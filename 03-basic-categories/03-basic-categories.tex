\documentclass[11pt, oneside]{article}

\usepackage{geometry}
\geometry{letterpaper}
\usepackage{tipa,vowel,gb4e}
\let\ipa\textipa
\def\asp{\ipa{\super h}}
\let\eachwordone=\it

\usepackage{multirow}

\title{Basic Categories of the Ishiculu language}
\author{Ming Zhang}
\date{September 22, 2117}

\begin{document}
\maketitle

Ishiculu has nouns, verbs, adjectives, prepositions, and conjunction.

Examples of sentences in Ishiculu:

\begin{exe}
\ex
\gll Ni-ki-so \ipa{nta\textbeltl o} kitabu. \\
\textsc{1sg}-\textsc{3sg}-read \textsc{clf}.vii book(vii) \\
\trans `I (will) read the book.' \\
(vii = Noun Class 7)
\ex
\gll Ni-\ipa{\textbeltl}a-mbi Elly ki-ke \ipa{nta\textbeltl o} kitabu. \\
\textsc{1sg}-\textsc{caus}-\textsc{pst} Elly \textsc{1sg}-have \textsc{clf}.vii book(vii) \\
\trans `I gave Elly the book.'
\ex
\gll Elly pa ki-joani \ipa{nta\textbeltl o} kitabu. \\
Elly not \textsc{3sg}-like \textsc{clf}.vii book(vii) \\
\trans `Elly doesn't like the book.'
\ex
\gll Elly \ipa{ki-\textlyoghlig ats7} Philadelphia. \\
Elly \textsc{3sg}-come\_from Philadelphia \\
\trans `Elly is from Philadelphia.'
\ex
\gll Ni-shi-ki-\ipa{\textbeltl}a-mbi Elly fu-Philadelphia ki-hi \ipa{nta\textbeltl o} kitabu. \\
\textsc{1sg}-read\_aloud-\textsc{3sg}-\textsc{caus}-\textsc{pst} Elly in-Philadelphia \textsc{3sg}-{hear} \textsc{clf}.vii book(vii) \\
\trans `I read the book aloud to Elly.'
\end{exe}

Vocabulary list:
\begin{itemize}
\item Nouns: \textit{kitabu} `book', \textit{mama} `mother'
\item Verbs: \textit{\ipa{nta\textbeltl o}} `read', \textit{shi} `read aloud'
\item Adjectives: \textit{tata} `big', \textit{\ipa{\textbeltl o\textbeltl o} }`small'
\item Prepositions: \textit{fu-} `in, at', \textit{\ipa{ts7}} `out of'
\item Conjunctions: \textit{aha} `and', \textit{alo} `but'
\end{itemize}

\end{document}