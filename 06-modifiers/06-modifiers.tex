\documentclass[12pt, oneside]{article}

\usepackage[margin=1.2in]{geometry}
\geometry{letterpaper}
\usepackage{tipa,vowel,gb4e}
\let\ipa\textipa
\let\eachwordone=\it

\usepackage{multirow}

\title{Modifiers of the Ishiculu language}
\author{Ming Zhang}
\date{October 20, 2117}

\begin{document}
\maketitle

Some glossing abbreviations used in this chapter: 

\begin{enumerate}
\item \textbf{Verb}-Object
\begin{exe}
\ex
\gll Tse-kiw-ioani-mbi nta\textbeltl o kitabu. \\
\textsc{1pl}-\textsc{3sg}-like-\textsc{pst} \textsc{clf}.7 book(7) \\
\trans `We used to like the book.'
\end{exe}
\item Classifier-\textbf{Noun}
\begin{exe}
\ex
\gll i {ntoch\textramshorns} \\
\textsc{clf.8} car(8) \\
\trans `the car(s)'
\end{exe}
\item \textbf{Noun}-Adjective
\begin{exe}
\ex
\gll {ntoch\textramshorns} i-caca \\
car(8) 8-big \\
\trans `big car(s)'
\end{exe}
\item Numeral-Classifier-\textbf{Noun}

A numeral modifying a noun requires a classifier, which follows the numeral.
\begin{exe}
\ex
\gll nta nta\textbeltl o kitabu \\
three \textsc{clf.7} book(7) \\
\trans `three books'
\end{exe}
\item Demonstrative-\textbf{Noun}
\begin{exe}
\ex
\gll laba nta\textbeltl o kitabu \\
these \textsc{clf.7} book(7) \\
\trans `these books'
\end{exe}
\item \textbf{Noun}-Genitive
\begin{exe}
\ex
\gll nta\textbeltl o kitabu ce-mama \\
\textsc{clf.7} book(7) \textsc{gen}-mother \\
\trans `mother's book(s)'
\end{exe}
\item Preposition-\textbf{Noun}-Postposition

Note that the presence of both prepositions and postpositions is uncommon, especially among VO languages.
\begin{exe}
\ex
\gll fu Philadelphia tso \\
\textsc{prep.int} Philadelphia \textsc{post.abl} \\
\trans `from Philadelphia'
\end{exe}
\item Preposition-Demonstrative-Numeral-Classifier-\textbf{Noun}-Adjective-Possessor-Postposition
\begin{exe}
\ex
\gll fu laba nta u conkei u-caca ce-John tso \\
\textsc{prep.int} these three \textsc{clf.4} house 4-big \textsc{gen}-John \textsc{post.abl} \\
\trans `out of these three big houses of John'
\ex
\gll Fu u conkei u-\textbeltl oko tse-ke-\textbeltl a-ki Ari nta nta\textbeltl o kitabu ce-Mel \\
\textsc{prep.int} \textsc{clf.4} house 4-red \textsc{1pl}-have-\textsc{caus}-\textsc{3sg} Ari three \textsc{clf}.7 book(7) \textsc{poss}-Mel \\
\trans `We will give Ari Mel's three books in the red house.'
\end{exe}

\end{enumerate}


\end{document}