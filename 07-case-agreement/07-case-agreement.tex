\documentclass[12pt, oneside]{article}

\usepackage[margin=1.2in]{geometry}
\geometry{letterpaper}
\usepackage{tipa,vowel,gb4e}
\let\ipa\textipa
%\let\eachwordone=\it

\usepackage{multirow}

\title{Case/Agreement of the Ishiculu language}
\author{Ming Zhang}
\date{October 27, 2117}

\begin{document}
\maketitle

Ishiculu has a rich agreement system on verbs and does not exhibit a case system. In this sense, it is predominantly head-marking.

\begin{exe}
\ex
\gll \textit{Shi-shiy-ioani-\o} \ipa{nta\textbeltl o} kitabu. \\
\textsc{1pl}-\textsc{cl7.sg}-like-\textsc{prs} \textsc{clf}.7 book(7) \\
\trans `We (will) like the book.'
\end{exe}

Ishiculu follows accusative alignment. The verb is marked according to a nominative-accusative system. The verb stem is inflected based on the person, number, and for 3rd person nominal phrases, the noun class.

\begin{center}
\begin{tabular}{c|c|c|c|c}
\hline
\multirow{2}{*}{Sbj/obj prefixes} & \multicolumn{2}{c|}{singular} & \multicolumn{2}{c}{plural} \\
\cline{2-5}
 & before C & before V & before C & before V \\
\hline
\hline
1st person & -ngi- & -ngiw- & -shi- & -shiy- \\
\hline
2nd person & -u- & -uw- & -ni-& -niy- \\
\hline
CL 1/2 & \multicolumn{2}{c|}{-ka-} & \multicolumn{2}{c}{-ba-} \\
\hline
CL 3/4 & -u- & -uw- & -i- & -iy- \\
\hline
CL 5/6 & -\textbeltl i- & -\textbeltl iy- & \multicolumn{2}{c}{-a-} \\
\hline
CL 7/8 & -shi- & -shiy- & -\ipa{Z}i- & -\ipa{Z}iy- \\
\hline
\end{tabular}
\end{center}

Ishiculu pronouns also exhibit no case morphology.

\begin{center}
\begin{tabular}{c|c|c}
\hline
Pers. pron. & singular & plural \\
\hline
\hline
1st person & mina & thina \\
\hline
2nd person & wena & nina \\
\hline
CL 1/2 & yena & bona \\
\hline
CL 3/4 & wona & yona \\
\hline
CL 5/6 & lona & wona \\
\hline
CL 7/8 & sona & \ipa{Z}ona \\
\hline
\end{tabular}
\end{center}

\end{document}