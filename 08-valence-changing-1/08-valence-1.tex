\documentclass[12pt, oneside]{article}

\usepackage[margin=1.2in]{geometry}
\geometry{letterpaper}
\usepackage{tipa,vowel,gb4e}
\let\ipa\textipa
%\let\eachwordone=\it

\usepackage{multirow}

\title{Valence-Changing Operations of the Ishiculu language}
\author{Ming Zhang}
\date{November 6, 2117}

\begin{document}
\maketitle

Ishiculu has a rich agreement system on verbs and does not exhibit a case system. In this sense, it is predominantly head-marking.

\section{Passive}

Morphological passive in Ishiculu is marked with \textit{\ipa{-wal7}} and decreases the valence of transitive verbs.

\begin{exe}
\ex
\gll Nta\textbeltl o kitabu shi-so-\textit{\ipa{wal7}}-mbi. \\
\textsc{clf}.7 book(7) \textsc{7}-read-\textsc{pass}-\textsc{pst}  \\
\trans `The book was read.'
\ex
\gll U -bon -\textit{\ipa{wal7}} -mbi. \\
\textsc{2sg} -see -\textsc{pass} -\textsc{pst}  \\
\trans `You were seen.'
\end{exe}

\section{Impersonal}
When the verb is impersonal, it does not bear a subject concord prefix.

Ishiculu allows for impersonal meteorological predicates.

\begin{exe}
\ex
\gll Cofi ha-mbi. \\
yesterday rain-\textsc{pst} \\
\trans `It rained yesterday.'
\ex
\gll T\ipa{um7} \ipa{r7}. \\
tomorrow hot \\
\trans `It will be hot tomorrow.'
\end{exe}

Impersonal construction is also used for indefinite subjects.

\begin{exe}
\ex
\gll Pa ngi-shi-so nta\textbeltl o kitabu. \\
not \textsc{1sg}-\textsc{7}-read \textsc{clf}.7 book \\
\trans `I don't read the book.'
\ex
\gll Pa shi-so nta\textbeltl o kitabu. \\
not \textsc{7}-read \textsc{clf}.7 book \\
\trans `One doesn't/shouldn't read the book.'
\end{exe}

\section{Causative}
The suffix \textit{-\textbeltl a} marks the causative form of a verb. The causer of the causative verb is marked with an additional prefix.

\begin{exe}
\ex
\gll Ngi-ka-shi-so-\ipa{\textbeltl}a Ari \ipa{nta\textbeltl o} kitabu. \\
\textsc{1sg}-\textsc{1}-\textsc{7}-have-\textsc{caus} Ari \textsc{clf}.7 book(7) \\
\trans `I (will) make Ari read the book.'
\ex
\gll Fu-Philadelphia ngi-ka-shi-ke-\ipa{\textbeltl}a Ari \ipa{nta\textbeltl o} kitabu. \\
in-Philadelphia \textsc{1sg}-\textsc{1}-\textsc{7}-have-\textsc{caus} Ari \textsc{clf}.7 book(7) \\
\trans `I (will) give the book to Ari in Philadelphia.'
\end{exe}

\end{document}