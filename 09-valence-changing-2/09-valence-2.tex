\documentclass[12pt, oneside]{article}

\usepackage[margin=1.2in]{geometry}
\geometry{letterpaper}
\usepackage{tipa,vowel,gb4e}
\let\ipa\textipa
%\let\eachwordone=\it

\usepackage{multirow}

\title{Valence-Changing Operations of the Ishiculu language}
\author{Ming Zhang}
\date{November 13, 2117}

\begin{document}
\maketitle

\section{Causative}

The suffix \textit{-\textbeltl a} marks the causative form of a verb. The causer of the causative verb is marked with an additional prefix on the verb.

\begin{exe}
\ex
\gll Ngi-ka-shi-so-\ipa{\textbeltl}a Ari \ipa{nta\textbeltl o} kitabu. \\
\textsc{1sg}-\textsc{1}-\textsc{7}-have-\textsc{caus} Ari \textsc{clf}.7 book(7) \\
\trans `I (will) make Ari read the book.'
\end{exe}

\subsection{Ditransitive}

Prototypical verbs in Ishiculu are transitive at most, and verbs that semantically require three thematic core arguments are usually expressed through causatives of transitive verbs.

\begin{exe}
\ex
\gll Fu-Philadelphia ngi-ka-shi-ke-\ipa{\textbeltl}a Ari \ipa{nta\textbeltl o} kitabu. \\
in-Philadelphia \textsc{1sg}-\textsc{1}-\textsc{7}-have-\textsc{caus} Ari \textsc{clf}.7 book(7) \\
\trans `I (will) give Ari the book in Philadelphia.'
\ex
\gll {W\textramshorns} babamama ba-ngiw-u-nde-\ipa{\textbeltl}a-mbi mina shizi. \\
\textsc{clf.2} parents(2) \textsc{2.pl}-\textsc{1sg}-\textsc{4.sg}-eat-\textsc{caus}-\textsc{pst} \textsc{1sg} cheese(4) \\
\trans `My parents fed me cheese.'
\end{exe}

\section{Passive of causative}

The causee or the object can be promoted to the subject position with the passive construction. Following the parallelism between ditransitive and causative verbs, their passive constructions are also indistinguishable in morphology.

\subsection{Promoting causee}

This refers to the passive construction similar to the following:

\begin{exe}
\ex Passive of causative of intransitive in English \\
John makes me cringe. $\rightarrow$ I am made to cringe (by John).
\ex Passive of causative of transitive in English \\
John makes the computer change the date. $\rightarrow$ The computer is made to change the date (by John).
\ex Passive of ditransitive in English \\
John gives the teacher the book. $\rightarrow$ The teacher is given the book (by John).
\end{exe}

This construction in Ishiculu is formed by a passive suffix \textit{-\ipa{wal7}} after the causative suffix. The personal suffix of the causer on the verb is then dropped.

\begin{exe}
\ex Passive of causative of intransitive in Ishiculu
\begin{xlist}
\ex
\gll Ngi-ka -hulu-\textbeltl a-mbi John. \\
\textsc{1sg}-\textsc{1.sg} -cry-\textsc{caus}-\textsc{pst} John \\
\trans `I made John cry.'
\ex
\gll John ka -hulu-\textbeltl a-wal\textramshorns-mbi. \\
John \textsc{1.sg} -cry-\textsc{caus}-\textsc{pass}-\textsc{pst} \\
\trans `John was made to cry.'
\end{xlist}
\ex Passive of ditransitive/causative of transitive in Ishiculu
\begin{xlist}
\ex
\gll Shiy-u-ko -me\textbeltl i-\textbeltl a-mbi hehe. \\
\textsc{1pl}-\textsc{2sg}-\textsc{5.sg} -receive\_by\_mail-\textsc{caus}-\textsc{pst} cake(5) \\
\trans `We mailed you cake.'
\ex
\gll U-ko -me\textbeltl i-\textbeltl a-wal\textramshorns-mbi hehe. \\
\textsc{2sg}-\textsc{5.sg} -receive\_by\_mail-\textsc{caus}-\textsc{pass}-\textsc{pst} cake(5) \\
\trans `You were mailed cake.'
\end{xlist}
\end{exe}

\subsection{Promoting object}

There is no passivization of a causative verb that directly promotes the underlying object, but it is possible to passivize the underlying verb, promoting the object to the causee position before promoting it further to the subject position of the matrix clause. As the subject in the active voice becomes impossible to be salient in the passive, the causee becomes ineffable in this passive construction.

\begin{exe}
\ex
\begin{xlist}
\ex Active voice
\gll John ka-u-shi -ke-\textbeltl a meyi sh\textramshorns nta\textbeltl o kitabu. \\
John \textsc{1.sg}-\textsc{3.sg}-\textsc{7.sg} -have-\textsc{caus} \textsc{clf.3} teacher(3) \textsc{clf.7} book(7) \\
\trans `John gives the teacher the book.'
\ex Impossibility of direct promotion of object
\gll * Nta\textbeltl o kitabu shi-u {-ke-\textbeltl a-wal\textramshorns} meyi sh\textramshorns. \\
{} \textsc{clf.7} book(7) \textsc{7.sg}-\textsc{3.sg} -have-\textsc{caus}-\textsc{pass} \textsc{clf.3} teacher(3) \\
\trans Intended meaning: `The book is given to the teacher.'
\ex Causative of passive
\gll John ka-u-shi -ke-wal\textramshorns-\textbeltl a nta\textbeltl o kitabu. \\
John \textsc{1.sg}-\textsc{3.sg}-\textsc{7.sg} -have-\textsc{pass}-\textsc{caus} \textsc{clf.7} book(7) \\
\trans `John gives the book.' (lit. `John makes the book be given.')
\ex Passive of causative of passive
\gll Nta\textbeltl o kitabu shi-ke-wal\textramshorns-\textbeltl a-wal\textramshorns. \\
\textsc{clf.7} book(7) \textsc{7.sg}-have-\textsc{pass}-\textsc{caus}-\textsc{pass} \\
\trans `The book is given.' (lit. `The book is made to be had.')
\ex Ineffability of the causee
\gll * Nta\textbeltl o kitabu shi-u {-ke-wal\textramshorns-\textbeltl a-wal\textramshorns} meyi sh\textramshorns. \\
{} \textsc{clf.7} book(7) \textsc{7.sg}-\textsc{3.sg} -have-\textsc{pass}-\textsc{caus}-\textsc{pass} \textsc{clf.3} teacher(3) \\
\trans Intended meaning: `The book is given to the teacher.' (lit. `The book is made to be had.')
\end{xlist}
\end{exe}

\end{document}