\documentclass[12pt, oneside]{article}

\usepackage[margin=1in]{geometry}
\geometry{letterpaper}
\usepackage{tipa,vowel,gb4e}
\let\ipa\textipa
%\let\eachwordone=\it

\usepackage{multirow}

\title{Questions and Answers of the Ishiculu Language}
\author{Ming Zhang}
\date{November 24, 2117}

\begin{document}
\maketitle

\section{Polar questions}

Polar questions are formed in Ishiculu with the polar question particle \textit{\ipa{le}} at the end of the sentence.

\begin{exe}
\ex
\begin{xlist}
\ex
\gll Meyi {sh\textramshorns} u-ka-shi-ke-\textbeltl a-mbi Ari nta\textbeltl o kitabu. \\
\textsc{clf.3} teacher(3) \textsc{3.sg}-\textsc{1.sg}-\textsc{7.sg}-have-\textsc{caus}-\textsc{pst} Ari \textsc{clf.7} book(7) \\
\trans `The teacher gave Ari the book.'
\ex
\gll Meyi {sh\textramshorns} u-ka-shi-ke-\textbeltl a-mbi Ari nta\textbeltl o kitabu le? \\
\textsc{clf.3} teacher(3) \textsc{3.sg}-\textsc{1.sg}-\textsc{7.sg}-have-\textsc{caus}-\textsc{pst} Ari \textsc{clf.7} book(7) Q \\
\trans `Did the teacher give Ari the book?'
\end{xlist}
\end{exe}

\section{Response to polar questions}

Ishiculu uses particles \textit{\ipa{yebo}}, `yes' and \textit{\ipa{cha}}, `no' to respond to polar questions. Ishiculu's response strategy is also truth-based; that is, \textit{\ipa{cha}} confirms a negative question, and \textit{\ipa{yebo}} contradicts the negation of a negative question.

\begin{exe}
\ex
\begin{xlist}
\ex
\gll Meyi {sh\textramshorns} u-ka-shi-ke-\textbeltl a-mbi Ari nta\textbeltl o kitabu le? \\
\textsc{clf.3} teacher(3) \textsc{3.sg}-\textsc{1.sg}-\textsc{7.sg}-have-\textsc{caus}-\textsc{pst} Ari \textsc{clf.7} book(7) Q \\
\trans `Did the teacher give Ari the book?'
\ex
\gll Yebo. \\
yes \\
\trans `Yes(, the teacher gave Ari the book).'
\ex
\gll Cha. \\
no \\
\trans `No(, the teacher did not give Ari the book).'
\end{xlist}
\ex
\begin{xlist}
\ex
\gll Meyi {sh\textramshorns} pa u-ka-shi-ke-\textbeltl a-mbi Ari nta\textbeltl o kitabu le? \\
\textsc{clf.3} teacher(3) not \textsc{3.sg}-\textsc{1.sg}-\textsc{7.sg}-have-\textsc{caus}-\textsc{pst} Ari \textsc{clf.7} book(7) Q \\
\trans `Did the teacher not give Ari the book?'
\ex
\gll Yebo. \\
yes \\
\trans `Yes(, the teacher gave Ari the book).'
\ex
\gll Cha. \\
no \\
\trans `No(, the teacher did not give Ari the book).'
\end{xlist}
\end{exe}

\section{Content questions}

Content questions are formed with the question particle \textit{\ipa{le}} at the end of the sentence. Some WH-words in Ishiculu include \textit{\ipa{uba}}(2), `who;' \textit{\ipa{uni}}(4), `what;' \textit{\ipa{una}}, `where.'

When the WH-word is the subject, the question is formed with the WH-word \textit{in situ}.

\begin{exe}
\ex
\begin{xlist}
\ex
\gll {W\textramshorns} babamama ba-uw-u-nde-\ipa{\textbeltl}a-mbi wena shizi. \\
\textsc{clf.2} parents(2) \textsc{2.pl}-\textsc{2sg}-\textsc{4.sg}-eat-\textsc{caus}-\textsc{pst} \textsc{2sg} cheese(4) \\
\trans `Your parents fed you cheese.'
\ex
\gll Uba ka-uw-u-nde-\ipa{\textbeltl}a-mbi wena shizi le? \\
who(2) \textsc{2.sg}-\textsc{2sg}-\textsc{4.sg}-eat-\textsc{caus}-\textsc{pst} \textsc{2sg} cheese(4) Q \\
\trans `Who fed you cheese?'
\end{xlist}
\end{exe}

Non-subject WH-words need to be moved to the post-verbal position.

\begin{exe}
\ex
\gll {W\textramshorns} babamama ba-ka-u-nde-\ipa{\textbeltl}a-mbi uba shizi le? \\
\textsc{clf.2} parents(2) \textsc{2.pl}-\textsc{2.sg}-\textsc{4.sg}-eat-\textsc{caus}-\textsc{pst} who cheese(4) Q \\
\trans `To whom did your parents feed cheese?'
\ex
\gll {W\textramshorns} babamama ba-uw-u-nde-\ipa{\textbeltl}a-mbi uni wena le? \\
\textsc{clf.2} parents(2) \textsc{2.pl}-\textsc{4.sg}-\textsc{4.sg}-eat-\textsc{caus}-\textsc{pst} what \textsc{2sg} Q \\
\trans `What did your parents feed you?'
\ex
\gll {W\textramshorns} babamama ba-uw-u-nde-\ipa{\textbeltl}a-mbi una wena shizi le? \\
\textsc{clf.2} parents(2) \textsc{2.pl}-\textsc{2sg}-\textsc{4.sg}-eat-\textsc{caus}-\textsc{pst} where \textsc{2sg} cheese(4) Q \\
\trans `Where did your parents feed you cheese?'
\end{exe}

When there are more than one WH-word, one of them is regarded as the primary WH-word, which follows the WH-word movement described above. The remaining WH-words are \textit{in situ}.

\begin{exe}
\ex
\begin{xlist}
\ex
\gll Uba ka-uw-u-nde-\ipa{\textbeltl}a-mbi wena uni le? \\
who \textsc{2.sg}-\textsc{2sg}-\textsc{4.sg}-eat-\textsc{caus}-\textsc{pst} \textsc{2sg} what Q \\
\trans `Who fed you what?'
\ex
\gll Uba ka-uw-u-nde-\ipa{\textbeltl}a-mbi uni wena le? \\
who \textsc{2.sg}-\textsc{2sg}-\textsc{4.sg}-eat-\textsc{caus}-\textsc{pst} what \textsc{2sg} Q \\
\trans `What did who feed you?'
\end{xlist}
\end{exe}


\end{document}