\documentclass[12pt, oneside]{article}

\usepackage[margin=1in]{geometry}
\geometry{letterpaper}
\usepackage{tipa,vowel,gb4e}
\let\ipa\textipa
%\let\eachwordone=\it

\usepackage{multirow}

\title{Classifiers of the Ishiculu Language}
\author{Ming Zhang}
\date{December 1, 2117}

\begin{document}
\maketitle

Ishiculu employs classifiers to company nouns when they refer to entities in a non-generic ways. The classifier modifying a noun is determined by the class of the modified noun.

\begin{center}
\begin{tabular}{c|c}
\hline
noun class & classifier \\
\hline
1 & wo \\
\hline
2 & w\textramshorns \\
\hline
3 & meyi \\
\hline
4 & u \\
\hline
5 & ishi \\
\hline
6 & i\textlyoghlig i \\
\hline
7 & nta\textbeltl o \\
\hline
8 & a\textlyoghlig o \\
\hline
\end{tabular}
\end{center}

When the noun is used generically instead of to refer to its instances, we don't use classifiers.

\begin{exe}
\ex
\gll Ngi-u-nde-mbi shidzi. \\
\textsc{1sg}-\textsc{4.sg}-eat-\textsc{pst} cheese(4) \\
\trans ``I ate cheese.''
\end{exe}

When the noun is specified, a classifier will need to accompany it.

\begin{exe}
\ex
\gll Ngi-u-nde-mbi \textit{u} shidzi \textit{ce-John}. \\
\textsc{1sg}-\textsc{4.sg}-eat-\textsc{pst} \textsc{clf.4} cheese(4) \textsc{gen}-John \\
\trans ``I ate John's cheese.''
\ex
\gll Ngi-u-nde-mbi \textit{ligha} \textit{u} shidzi. \\
\textsc{1sg}-\textsc{4.sg}-eat-\textsc{pst} this \textsc{clf.4} cheese(4) \\
\trans ``I ate this cheese.''
\ex
\gll Ngi-u-nde-mbi \textit{u} shidzi \textit{uw-uw-ioani} \textit{wona}. \\
\textsc{1sg}-\textsc{4.sg}-eat-\textsc{pst} \textsc{clf.4} cheese(4) \textsc{2sg}-\textsc{4.sg}-like \textsc{prn.4.sg} \\
\trans ``I ate the cheese that you like.''
\end{exe}

A classifier is required when a numeral modifies the noun.

\begin{exe}
\ex
\gll Ngi-zhi-ke \textit{nta} \textit{nta\textbeltl o} kitabu. \\
\textsc{1sg}-\textsc{7.pl}-have three \textsc{clf.7} book \\
\trans ``I have three books.''
\end{exe}

On WALS, there are 400 languages surveyed in the feature of numeral classifiers, and 140 of them have classifiers accompanying nouns when numerals are used. Among them, the three Chinese languages surveyed, Mandarin, Cantonese, and Hokkien all have numeral classifiers. Even though in traditional grammar, Zulu is not considered to have any classifier, the noun classes in Zulu fit nicely with the classifiers coming from Chinese languages, especially from Cantonese, where classifiers are more versatile, not just used with numerals.

\end{document}