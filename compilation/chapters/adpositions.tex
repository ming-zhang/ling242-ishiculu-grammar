\chapter{Adpositions}
\newthought{Ishiculu nouns do not exhibit} case markings but rather receive prepositions and postpositions. Prepositions are used to mark locative case, and overt postpositions are only used in marking the lative or the ablative, shown in \ref{table:adpositions:adpositions}. Note that the presence of both prepositions and postpositions is uncommon, especially among VO languages \cite{wals-85}.

\begin{table}
\centering
\begin{tabular}{l|l|l|l}
\hline
 & \textit{\o}: static & \textit{tso}: ablative & \textit{ghu}: lative \\
\hline
\textit{fu}: interior & `in, at, inside' & `out of' & `into' \\
\hline
\textit{ts\textipa{7}}: surface & `on the surface of' & `off the surface of' & `onto the surface of' \\
\hline
\end{tabular}
\caption{Prepositions and postpositions in Ishiculu.}
\label{table:adpositions:adpositions}
\end{table}

\begin{exe}
\ex
\gll Fu Philadelphia {\o} shi-ka-ke-\textipa{\textbeltl}a Ari \textipa{nta\textbeltl o} kitabu. \\
\textsc{prep.int} Philadelphia \textsc{post.sta} \textsc{1pl}-\textsc{cl1.sg}-have-\textsc{caus} Ari \textsc{clf}.7 book(7) \\
\trans `We will give Ari the book in Philadelphia.'
\ex
\gll Ari fu Philadelphia tso. \\
Ari \textsc{prep.int} Philadelphia \textsc{post.abl} \\
\trans `Ari is from Philadelphia.'
\end{exe}