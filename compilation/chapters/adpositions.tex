\chapter{Adpositions}


Ishiculu nouns do not exhibit case markings but rather receive prepositions for the location of the locative case, and postpositions for the direction of any movement involved.

\begin{center}
\begin{tabular}{l|l|l|l}
\hline
 & \textit{-\o}: static & \textit{ts\textipa{O}}: centrifugal & \textit{\textipa{Zu}}: centripetal \\
\hline
\textit{fu}: interior & `in, at, inside' & `out of' & `into' \\
\hline
\textit{ts\textipa{7}}: surface & `on the surface of' & `off the surface of' & `onto the surface of' \\
\hline
\end{tabular}
\end{center}

\begin{exe}
\ex
\gll Fu Philadelphia {\o} tse-ke-\textipa{\textbeltl}a-ki Ari \textipa{nta\textbeltl o} kitabu. \\
\textsc{prep.int} Philadelphia \textsc{post.sta} \textsc{1sg}-have-\textsc{caus}-\textsc{3sg} Ari \textsc{clf}.7 book(7) \\
\trans `We will give Ari the book in Philadelphia.'
\ex
\gll Ari fu-Philadelphia-tso. \\
Ari \textsc{int}-Philadelphia-\textsc{centf} \\
\trans `Ari is from Philadelphia.'
\end{exe}