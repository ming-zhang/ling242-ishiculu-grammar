\chapter{Classifiers} \label{ch:classifiers}

\newthought{Approximately 35\%} of world languages have numeral classifiers, among which are Mandarin and Cantonese \cite{wals-55}. Even though in traditional grammar, Zulu is not considered to have any classifier, the noun classes in Zulu fit nicely with the classifiers coming from Chinese languages, especially from Cantonese, where classifiers are more versatile outside of contexts with numerals. In Ishiculu, the classifier modifying a noun is determined by the class of the modified noun, shown in Table~\ref{table:classifiers:clf}

\begin{table}
\centering
\begin{tabular}{c|c}
\hline
noun class & classifier \\
\hline
1 & wo \\
\hline
2 & w\textramshorns \\
\hline
3 & meyi \\
\hline
4 & u \\
\hline
5 & ishi \\
\hline
6 & i\textlyoghlig i \\
\hline
7 & nta\textbeltl o \\
\hline
8 & a\textlyoghlig o \\
\hline
\end{tabular}
\caption{Classifiers in Ishiculu.}
\label{table:classifiers:clf}
\end{table}

When the noun is used generically instead of as reference to its instances, no classifier is used.

\begin{exe}
\ex
\gll Ngi-u-nde-mbi shidzi. \\
\textsc{1sg}-\textsc{cl4.sg}-eat-\textsc{pst} cheese(4) \\
\trans `I ate cheese.'
\end{exe}

A classifier is required when a numeral modifies the noun.

\begin{exe}
\ex
\gll Ngi-zhi-ke \textit{nta} \textit{nta\textbeltl o} kitabu. \\
\textsc{1sg}-\textsc{cl7.pl}-have three \textsc{clf.7} book \\
\trans `I have three books.'
\end{exe}

Ishiculu employs classifiers to accompany nouns when they refer to entities in a non-generic ways. When the noun is specified in the following ways, a classifier is obligatory.

\begin{exe}
\ex Modified by a genitive
\gll Ngiw-u-nde-mbi \textit{u} shidzi \textit{ce-John}. \\
\textsc{1sg}-\textsc{cl4.sg}-eat-\textsc{pst} \textsc{clf.4} cheese(4) \textsc{gen}-John \\
\trans `I ate John's cheese.'
\ex Modified by a demonstrative
\gll Ngiw-u-nde-mbi \textit{ligha} \textit{u} shidzi. \\
\textsc{1sg}-\textsc{cl4.sg}-eat-\textsc{pst} this \textsc{clf.4} cheese(4) \\
\trans `I ate this cheese.'
\ex Modified by a relative clause
\gll Ngiw-u-nde-mbi \textit{u} shidzi \textit{uw-uw-ioani} \textit{wona}. \\
\textsc{1sg}-\textsc{cl4.sg}-eat-\textsc{pst} \textsc{clf.4} cheese(4) \textsc{2sg}-\textsc{cl4.sg}-like \textsc{prn.cl4.sg} \\
\trans `I ate the cheese that you like.'
\end{exe}

