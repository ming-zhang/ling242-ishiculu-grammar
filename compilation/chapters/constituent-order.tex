\chapter{Constituent order}

\section{Basic word order}
\newthought{Ishiculu is subject-prominent}, with a basic word order of SVO.

\begin{exe}
\ex
\gll Ari pa ka-shiy-ioani nta\textbeltl o kitabu. \\
Ari not \textsc{cl1.sg}-\textsc{cl7.sg}-like \textsc{clf}.7 book(7) \\
\trans `Ari doesn't like the book.'
\ex
\gll Shi-shiy-ioani-mbi nta\textbeltl o kitabu. \\
\textsc{1pl}-\textsc{cl7.sg}-like-\textsc{pst} \textsc{clf}.7 book(7) \\
\trans `We used to like the book.'\ex
\gll Fu Philadelphia shi-ka-shi-ke-\textbeltl a Ari nta\textbeltl o kitabu. \\
\textsc{prep.int} Philadelphia \textsc{1pl}-\textsc{cl1.sg}-\textsc{cl7.sg}-have-\textsc{caus} Ari \textsc{clf}.7 book(7) \\
\trans `We will give Ari the book in Philadelphia.'
\end{exe}

Sometimes in Ishiculu, the predicate position can be filled by an uninflected adjectival phrase, such as an adjective or an adpositional phrase.

\begin{exe}
\ex
\gll Ari fu Philadelphia tso. \\
Ari \textsc{prep.int} Philadelphia \textsc{post.abl} \\
\trans `Ari is from Philadelphia.'
\ex
\gll Ari \textbeltl o\textbeltl o. \\
Ari small \\
\trans `Ari is little.'
\end{exe}

\subsection{Locative expression}
A locative phrase can occupy the subject position of an intransitive verb of existence or movement while the subject is then moved to the object position.

\begin{exe}
\ex
\gll {Ts\textramshorns} nta\textbeltl o nji tso \textbeltl i-huma nta\textbeltl o ika. \\
\textsc{prep.sur} \textsc{clf.7} table(7) \textsc{post.abl} \textsc{cl6.sg}-jump \textsc{clf}.6 cat(6) \\
\trans `Off the table jumps a cat.'
\ex
\gll Fu nta\textbeltl o heje shi-paka a\textlyoghlig o ntoch\textramshorns. \\
\textsc{prep.int} \textsc{clf.7} courtyard(7) \textsc{cl8.sg}-park \textsc{clf}.8 car(8) \\
\trans `In the courtyard is parked a car.'
\end{exe}

\subsection{Topicalization}
Topicalization can be carried out on the object to put emphasis, but a classifier for the object needs to remain in the post-verbal position.

\begin{exe}
\ex
\gll Kitabu shi-shi-so-mbi \textit{nta\textbeltl o}. \\
book(7) \textsc{1pl}-\textsc{cl7.sg}-read-\textsc{pst} \textsc{clf}.7 \\
\trans `The book, we have read it.'
\ex
\gll {Ntoch\textramshorns} ngi-shiy-ioani \textit{a\textlyoghlig o}. \\
car(8) \textsc{1sg}-\textsc{cl8.sg}-like \textsc{clf}.8 \\
\trans `The car, I like it.'
\end{exe}

\section{Modifiers of noun}

When demonstrative and/or numeral modify a noun, they always precede the noun. If they both modify the same noun, they are also found in this order. This particular word order is not uncommon among world's languages \cite{Greenberg-1963}. When a numeral modifies a noun, a classifier must also be used, which directly precedes the noun. See Chapter~\ref{ch:classifiers}.

\begin{exe}
\ex
\gll laba nta\textbeltl o kitabu \\
these \textsc{clf.7} book(7) \\
\trans `these books'
\ex
\gll a\textlyoghlig o {ntoch\textramshorns} \\
\textsc{clf.8} car(8) \\
\trans `the car(s)'
\ex
\gll nta nta\textbeltl o kitabu \\
three \textsc{clf.7} book(7) \\
\trans `three books'
\ex
\gll laba nta nta\textbeltl o kitabu \\
these three \textsc{clf.7} book(7) \\
\trans `these three books'
\end{exe}

An attributive adjective or a genitive follows the noun it modifies.

\begin{exe}
\ex
\gll {ntoch\textramshorns} caca \\
car big \\
\trans `big car(s)'
\ex
\gll nta\textbeltl o kitabu ce-mama \\
\textsc{clf.7} book(7) \textsc{gen}-mother \\
\trans `mother's book(s)'
\ex
\gll Fu u conkei \textbeltl oko shi-ka-shi-ke-\textbeltl a Ari nta nta\textbeltl o kitabu ce-Mel. \\
\textsc{prep.int} \textsc{clf.4} house(4) 4-red \textsc{1pl}-\textsc{cl1.sg}-\textsc{cl7.sg}-have-\textsc{caus} Ari three \textsc{clf}.7 book(7) \textsc{gen}-Mel \\
\trans `We will give Ari Mel's three books in the red house.'
\end{exe}

Example~\ref{ex:consitituent-order:all-modifiers} illustrates the ordering of all types of modifiers and adpositions with the same noun.

\begin{exe}
\ex
\glll fu laba nta u conkei caca ce-John tso \\
\textsc{prep.int} these three \textsc{clf.4} house big \textsc{gen}-John \textsc{post.abl} \\
Preposition Demonstrative Numeral Classifier \textbf{Noun} Adjective Genitive Postposition \\
\trans `out of these three big houses of John'
\label{ex:consitituent-order:all-modifiers}
\end{exe}