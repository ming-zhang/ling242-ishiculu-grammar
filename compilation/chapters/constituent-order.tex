\chapter{Constituent order}


Ishiculu is subject-prominent. The basic word order of Ishiculu is SVO. More accurately, its basic word order is subject-predicate, where the predicate can be a VP, an AdjP, or a PP.

Ishiculu uses both prepositions and postpositions. All prepositions are used to mark locative case, but overt postpositions are only used in marking lative and ablative.

\begin{exe}
\ex
\gll Ari pa kiw-ioani nta\textbeltl o kitabu. \\
Ari not \textsc{3sg}-like \textsc{clf}.7 book(7) \\
\trans `Ari doesn't like the book.'
\ex
\gll Tse-kiw-ioani-mbi nta\textbeltl o kitabu. \\
\textsc{1pl}-\textsc{3sg}-like-\textsc{pst} \textsc{clf}.7 book(7) \\
\trans `We used to like the book.'
\ex
\gll Ari fu-Philadelphia-tso. \\
Ari \textsc{prep.int}-Philadelphia-\textsc{post.abl} \\
\trans `Ari is from Philadelphia.'
\ex
\gll {Fu-Philadelphia-\o} tse-ke-\textbeltl a-ki Ari nta\textbeltl o kitabu. \\
\textsc{prep.int}-Philadelphia-\textsc{post.sta} \textsc{1pl}-have-\textsc{caus}-\textsc{3sg} Ari \textsc{clf}.7 book(7) \\
\trans `We will give Ari the book in Philadelphia.'
\end{exe}

A locative phrase can also occupy the subject position of verb and the subject will be moved to the object position.

\begin{exe}
\ex
\gll Ts\textramshorns-nji-tso ki-huma nta\textbeltl o ika. \\
\textsc{prep.sur}-table-\textsc{post.abl} \textsc{3sg}-jump \textsc{clf}.6 cat(6) \\
\trans `Off the table jumps a cat.'
\ex
\gll {Fu-heje-\o} ki-paka i ntoch\textramshorns. \\
\textsc{prep.int}-courtyard-\textsc{post.abl} \textsc{3sg}-park \textsc{clf}.8 car(8) \\
\trans `In the courtyard is parked a car.'
\end{exe}

To put the emphasis on the object, it is possible to have OSV word order, but there will need to be a classifier for the object remaining in the original position.

\begin{exe}
\ex
\gll Kitabu tse-ki-so-mbi nta\textbeltl o. \\
book(7) \textsc{1pl}-\textsc{3sg}-read-\textsc{pst} \textsc{clf}.7 \\
\trans `The book, we have read it.'
\ex
\gll {ntoch\textramshorns} ni-kiw-ioani i. \\
car(8) \textsc{1sg}-\textsc{3sg}-like \textsc{clf}.8 \\
\trans `The car, I like it.'
\end{exe}



\begin{enumerate}
\item \textbf{Verb}-Object
\begin{exe}
\ex
\gll Tse-kiw-ioani-mbi nta\textbeltl o kitabu. \\
\textsc{1pl}-\textsc{3sg}-like-\textsc{pst} \textsc{clf}.7 book(7) \\
\trans `We used to like the book.'
\end{exe}
\item Classifier-\textbf{Noun}
\begin{exe}
\ex
\gll i {ntoch\textramshorns} \\
\textsc{clf.8} car(8) \\
\trans `the car(s)'
\end{exe}
\item \textbf{Noun}-Adjective
\begin{exe}
\ex
\gll {ntoch\textramshorns} i-caca \\
car(8) 8-big \\
\trans `big car(s)'
\end{exe}
\item Numeral-Classifier-\textbf{Noun}

A numeral modifying a noun requires a classifier, which follows the numeral.
\begin{exe}
\ex
\gll nta nta\textbeltl o kitabu \\
three \textsc{clf.7} book(7) \\
\trans `three books'
\end{exe}
\item Demonstrative-\textbf{Noun}
\begin{exe}
\ex
\gll laba nta\textbeltl o kitabu \\
these \textsc{clf.7} book(7) \\
\trans `these books'
\end{exe}
\item \textbf{Noun}-Genitive
\begin{exe}
\ex
\gll nta\textbeltl o kitabu ce-mama \\
\textsc{clf.7} book(7) \textsc{gen}-mother \\
\trans `mother's book(s)'
\end{exe}
\item Preposition-\textbf{Noun}-Postposition

Note that the presence of both prepositions and postpositions is uncommon, especially among VO languages.
\begin{exe}
\ex
\gll fu Philadelphia tso \\
\textsc{prep.int} Philadelphia \textsc{post.abl} \\
\trans `from Philadelphia'
\end{exe}
\item Preposition-Demonstrative-Numeral-Classifier-\textbf{Noun}-Adjective-Possessor-Postposition
\begin{exe}
\ex
\gll fu laba nta u conkei u-caca ce-John tso \\
\textsc{prep.int} these three \textsc{clf.4} house 4-big \textsc{gen}-John \textsc{post.abl} \\
\trans `out of these three big houses of John'
\ex
\gll Fu u conkei u-\textbeltl oko tse-ke-\textbeltl a-ki Ari nta nta\textbeltl o kitabu ce-Mel \\
\textsc{prep.int} \textsc{clf.4} house 4-red \textsc{1pl}-have-\textsc{caus}-\textsc{3sg} Ari three \textsc{clf}.7 book(7) \textsc{poss}-Mel \\
\trans `We will give Ari Mel's three books in the red house.'
\end{exe}

\end{enumerate}

