\chapter{Embeddings}

\section{Coordination}

Nouns, verbs, and clauses all coordinate in Ishiculu. The coordination of nouns makes use of the conjunction \textit{\textipa{h7}}; that of verbs makes use of \textit{\textipa{futi}}; and clauses coordinate by juxtaposing.

\section{Relative clauses}

Relative clauses are restrictive in Ishiculu. They are externally headed and post-nominal. The canonical construction of the relative clauses relativizes the subject in the clause and does not include a complementizer or relative pronoun.

\begin{exe}
\ex
\gll Ngiw-uw-ioani meyi {sh\textramshorns} u-ka-shi-ke-\textbeltl a-mbi Ari nta\textbeltl o kitabu. \\
\textsc{1sg}-\textsc{3.sg}-like \textsc{clf.3} teacher(3) \textsc{3.sg}-\textsc{1.sg}-\textsc{7.sg}-have-\textsc{caus}-\textsc{pst} Ari \textsc{clf.7} book(7) \\
\trans `I like the teacher that gave Ari the book.'
\end{exe}

Direct objects cannot be relativized through the same construction. Rather, the speaker makes use of two strategies, the first of which is to passivize the verb so that the direct object is promoted to the subject position.

\begin{exe}
\ex
\gll * Ngiw-shi-ioani nta\textbeltl o kitabu meyi {sh\textramshorns} u-so-mbi. \\
{} \textsc{1sg}-\textsc{7.sg}-like \textsc{clf.7} book(7) \textsc{clf.3} teacher(3) \textsc{3.sg}-read-\textsc{pst} \\
\trans Intended meaning: `I like the book that the teacher read.'
\ex
\gll Ngiw-shi-ioani nta\textbeltl o kitabu u-so-wal\textramshorns-mbi. \\
\textsc{1sg}-\textsc{7.sg}-like \textsc{clf.7} book(7)  \textsc{3.sg}-read-\textsc{pass}-\textsc{pst} \\
\trans `I like the book that was read.'
\end{exe}

The other strategy is to use a resumptive pronoun in the relative clause.

\begin{exe}
\ex
\gll Ngiw-shi-ioani nta\textbeltl o kitabu meyi {sh\textramshorns} u-so-mbi \textit{sona}. \\
\textsc{1sg}-\textsc{7.sg}-like \textsc{clf.7} book(7) \textsc{clf.3} teacher(3) \textsc{3.sg}-read-\textsc{pst} \textsc{prn.7.sg} \\
\trans `I like the book that the teacher read.'
\end{exe}
