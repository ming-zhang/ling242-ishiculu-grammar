\chapter{Embeddings}

\section{Coordination}

\newthought{Nouns, verbs, and clauses} all coordinate in Ishiculu. The coordination of nouns makes use of the conjunction \textit{\textipa{h7}}. Verbs and clauses coordinate by juxtaposing.

\begin{exe}
\ex
\gll iwai {h\textramshorns} shidzi \\
wine and cheese \\
\trans `wine and cheese'
\ex
\gll Ngi-nde, ngi-hulu. \\
\textsc{1sg}-eat \textsc{1sg}-cry \\
\trans `I eat, and I cry.'
\end{exe}

\section{Relative clauses}

Relative clauses are restrictive in Ishiculu. They are externally headed and post-nominal and do not include a complementizer or relative pronoun.

The primary construction of the relative clauses relativizes the subject in the clause.

\begin{exe}
\ex
\gll Ngiw-uw-ioani meyi {sh\textramshorns} u-ka-shi-ke-\textbeltl a-mbi Ari nta\textbeltl o kitabu. \\
\textsc{1sg}-\textsc{cl3.sg}-like \textsc{clf.3} teacher(3) \textsc{cl3.sg}-\textsc{cl1.sg}-\textsc{cl7.sg}-have-\textsc{caus}-\textsc{pst} Ari \textsc{clf.7} book(7) \\
\trans `I like the teacher that gave Ari the book.'
\end{exe}

Direct objects cannot be relativized through the same construction. Rather, the speaker makes use of two strategies, the first of which is to passivize the relative clause so that the direct object is promoted to the subject position.

\begin{exe}
\ex
\gll * Ngi-shiy-ioani nta\textbeltl o kitabu meyi {sh\textramshorns} u-so-mbi. \\
{} \textsc{1sg}-\textsc{cl7.sg}-like \textsc{clf.7} book(7) \textsc{clf.3} teacher(3) \textsc{cl3.sg}-read-\textsc{pst} \\
\trans Intended meaning: `I like the book that the teacher read.'
\ex
\gll Ngi-shiy-ioani nta\textbeltl o kitabu u-so-wal\textramshorns-mbi. \\
\textsc{1sg}-\textsc{7.sg}-like \textsc{clf.7} book(7)  \textsc{3.sg}-read-\textsc{pass}-\textsc{pst} \\
\trans `I like the book that was read.'
\end{exe}

The other strategy is to include a resumptive pronoun in the relative clause.

\begin{exe}
\ex
\gll Ngi-shiy-ioani nta\textbeltl o kitabu meyi {sh\textramshorns} u-so-mbi \textit{sona}. \\
\textsc{1sg}-\textsc{7.sg}-like \textsc{clf.7} book(7) \textsc{clf.3} teacher(3) \textsc{3.sg}-read-\textsc{pst} \textsc{prn.7.sg} \\
\trans `I like the book that the teacher read.'
\end{exe}
