\chapter{Interrogatives}

\section{Polar questions}

\newthought{Polar questions} in Ishiculu are formed with the question particle \textit{\textipa{le}} at the end of the sentence.

\begin{exe}
\ex
\begin{xlist}
\ex
\gll Meyi {sh\textramshorns} u-ka-shi-ke-\textbeltl a-mbi Ari nta\textbeltl o kitabu. \\
\textsc{clf.3} teacher(3) \textsc{cl3.sg}-\textsc{cl1.sg}-\textsc{cl7.sg}-have-\textsc{caus}-\textsc{pst} Ari \textsc{clf.7} book(7) \\
\trans `The teacher gave Ari the book.' \\

\ 
\\

\ex
\gll Meyi {sh\textramshorns} u-ka-shi-ke-\textbeltl a-mbi Ari nta\textbeltl o kitabu le? \\
\textsc{clf.3} teacher(3) \textsc{cl3.sg}-\textsc{cl1.sg}-\textsc{cl7.sg}-have-\textsc{caus}-\textsc{pst} Ari \textsc{clf.7} book(7) Q \\
\trans `Did the teacher give Ari the book?'
\end{xlist}
\end{exe}

\subsection{Responding to polar questions}

Ishiculu uses particles \textit{\textipa{yebo}}, `yes' and \textit{\textipa{cha}}, `no' to respond to polar questions. Response strategy is also truth-based; i.e., \textit{\textipa{cha}} confirms a negative question, and \textit{\textipa{yebo}} contradicts the negation of a negative question.

\begin{exe}
\ex
\begin{xlist}
\ex
\gll Meyi {sh\textramshorns} u-ka-shi-ke-\textbeltl a-mbi Ari nta\textbeltl o kitabu le? \\
\textsc{clf.3} teacher(3) \textsc{cl3.sg}-\textsc{cl1.sg}-\textsc{cl7.sg}-have-\textsc{caus}-\textsc{pst} Ari \textsc{clf.7} book(7) Q \\
\trans `Did the teacher give Ari the book?'
\ex
\gll Yebo. \\
yes \\
\trans `Yes(, the teacher gave Ari the book).'
\ex
\gll Cha. \\
no \\
\trans `No(, the teacher did not give Ari the book).'
\end{xlist}

\ 
\\

\ex
\begin{xlist}
\ex
\gll Meyi {sh\textramshorns} pa u-ka-shi-ke-\textbeltl a-mbi Ari nta\textbeltl o kitabu le? \\
\textsc{clf.3} teacher(3) not \textsc{cl3.sg}-\textsc{cl1.sg}-\textsc{cl7.sg}-have-\textsc{caus}-\textsc{pst} Ari \textsc{clf.7} book(7) Q \\
\trans `Did the teacher not give Ari the book?'
\ex
\gll Yebo. \\
yes \\
\trans `Yes(, the teacher gave Ari the book).'
\ex
\gll Cha. \\
no \\
\trans `No(, the teacher did not give Ari the book).'
\end{xlist}
\end{exe}

\section{Content questions}

Content questions are formed with the question particle \textit{\textipa{le}} at the end of the sentence. Some WH-words in Ishiculu include \textit{\textipa{uba}}(2), `who;' \textit{\textipa{uni}}(4), `what;' \textit{\textipa{una}}, `where.'

When the WH-word is the subject, the question is formed with the WH-word \textit{in situ}.

\begin{exe}
\ex
\begin{xlist}
\ex
\gll {W\textramshorns} babamama ba-uw-u-nde-\textipa{\textbeltl}a-mbi wena shizi. \\
\textsc{clf.2} parents(2) \textsc{cl2.pl}-\textsc{2sg}-\textsc{cl4.sg}-eat-\textsc{caus}-\textsc{pst} \textsc{2sg} cheese(4) \\
\trans `Your parents fed you cheese.'
\ex
\gll Uba ka-uw-u-nde-\textipa{\textbeltl}a-mbi wena shizi le? \\
who(2) \textsc{cl2.sg}-\textsc{2sg}-\textsc{cl4.sg}-eat-\textsc{caus}-\textsc{pst} \textsc{2sg} cheese(4) Q \\
\trans `Who fed you cheese?'
\end{xlist}
\end{exe}

Non-subject WH-words need to be moved to the post-verbal position.

\begin{exe}
\ex
\gll {W\textramshorns} babamama ba-ka-u-nde-\textipa{\textbeltl}a-mbi \textit{uba} shizi le? \\
\textsc{clf.2} parents(2) \textsc{cl2.pl}-\textsc{cl2.sg}-\textsc{cl4.sg}-eat-\textsc{caus}-\textsc{pst} who(2) cheese(4) Q \\
\trans `To whom did your parents feed cheese?' \\

\ 
\\

\ex
\gll {W\textramshorns} babamama ba-uw-u-nde-\textipa{\textbeltl}a-mbi \textit{uni} wena le? \\
\textsc{clf.2} parents(2) \textsc{cl2.pl}-\textsc{cl4.sg}-\textsc{cl4.sg}-eat-\textsc{caus}-\textsc{pst} what(4) \textsc{2sg} Q \\
\trans `What did your parents feed you?'
\ex
\gll {W\textramshorns} babamama ba-uw-u-nde-\textipa{\textbeltl}a-mbi \textit{una} wena shizi le? \\
\textsc{clf.2} parents(2) \textsc{cl2.pl}-\textsc{2sg}-\textsc{cl4.sg}-eat-\textsc{caus}-\textsc{pst} where \textsc{2sg} cheese(4) Q \\
\trans `Where did your parents feed you cheese?'
\end{exe}

When there are more than one WH-word, one of them is regarded as the primary WH-word, which follows the WH-word movement strategies described above. The remaining WH-words are \textit{in situ}. In Example~\ref{ex:interrogatives:multiple-wh}, the WH-word in italics in the primary one.

\begin{exe}
\ex
\begin{xlist}
\ex
\gll \textit{Uba} ka-uw-u-nde-\textipa{\textbeltl}a-mbi wena uni le? \\
who(2) \textsc{cl2.sg}-\textsc{2sg}-\textsc{cl4.sg}-eat-\textsc{caus}-\textsc{pst} \textsc{2sg} what(4) Q \\
\trans `Who fed you what?'
\ex
\gll Uba ka-uw-u-nde-\textipa{\textbeltl}a-mbi \textit{uni} wena le? \\
who(2) \textsc{cl2.sg}-\textsc{2sg}-\textsc{cl4.sg}-eat-\textsc{caus}-\textsc{pst} what(4) \textsc{2sg} Q \\
\trans `What did who feed you?'
\end{xlist}
\label{ex:interrogatives:multiple-wh}
\end{exe}
