
\chapter{Demographics and ethnographics}

\newthought{Ishiculu is a creole language} spoken by the Kuhuch\textramshorns we, a community in KwaZulu-Natal, South Africa of people of mixed Zulu and Chinese ancestry. The term ``Kuhuch\textramshorns we'' comes from Zulu for ``mixed,'' and is the self-referent of the Kuhuch\textramshorns we people.

Before the 2020s, there had been American scholars and organizations in eastern South Africa building health infrastructure. This trend motivated other organizations, mostly from the Greater China Region, to add to their presence in eastern South Africa. By the end of the 2020s, because of a strengthened South African government and a pivot of US international policies, most Americans set out to contribute to cross-nation collaborations had left. In 2034, a private corporation from Taiwan discovered an oil well just off Richards Bay, a town in KwaZulu-Natal, South Africa. This incentivized more Chinese people to reside in eastern South Africa, and a community of Ishiculu speakers descended from a mixture of Zulu and Chinese people. By 2105, Ishiculu had about 1500 speakers in KwaZulu-Natal.

Ishiculu is now mostly spoken in KwaZulu-Natal near the east coast. In these communities where Ishiculu is spoken, Zulu and Chinese are usually also used. The Kuhuch\textramshorns we people usually live and engage in social interactions with Zulu and Chinese people. Both Chinese and Ishiculu have seen a slight decrease in the number of speakers since the 2080s, a fact possibly attributed partially to their lack of legal status. There has been some but insufficient literature on marginalization of the Kuhuch\textramshorns we people in both Zulu- and Chinese-speaking communities, but the existing research has indicated that the interactions between Ishiculu speakers and other peoples are mostly friendly and social, and that the marginalization stems from the way local educational and legal systems are set up. The major languages spoken in KwaZulu-Natal is listed in Table~\ref{table:intro:languages}, and the racial makeup of Richards Bay is shown in Table~\ref{table:intro:racial}.

\begin{table}
\centering
\begin{tabular}{l|l}
Zulu & 71.8\% \\
\hline
English & 15.2\% \\
\hline
Xhosa & 4.4\% \\
\hline
Afrikaans & 2.6\% \\
\hline
Ishiculu & $<$ 0.01\%
\end{tabular}
\caption{Major languages spoken in KwaZulu-Natal (2105)}
\label{table:intro:languages}
\end{table}

\begin{table}
\centering
\begin{tabular}{l|l}
Black African & 43.0\% \\
\hline
Coloured (Including Kuhuch\textramshorns we) & 4.2\% \\
\hline
Indian/Asian & 19.2\% \\
\hline
White & 33.1\% \\
\hline
Other & 0.4\%
\end{tabular}
\caption{Racial makeup of Richards Bay (2101)}
\label{table:intro:racial}
\end{table}

\begin{figure}
\begin{tikzpicture} 
    \node[anchor=south west,inner sep=0] (image) at (0,0) {\includegraphics[width=1\textwidth]{figures/world-map.png}};
    \begin{scope}[x={(image.south east)},y={(image.north west)}]
        \draw[->,red,ultra thick] (0.755,0.605) to [out=180,in=60] (0.49,0.22);
    \end{scope}
\end{tikzpicture}
\caption{Migration of Chinese people to South Africa in the 2030s}
\end{figure}