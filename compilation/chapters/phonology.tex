\chapter{Phonology}

\section{Phoneme inventory}

\newthought{There are 31 consonants} and six vowels in Ishiculu. This consonant-vowel ratio is considered moderately high crosslinguistically \cite{wals-3}.

\begin{table}
\centering
\begin{tabular}{|l|l|c|c|c|c|c|}
\hline
\multicolumn{2}{|c|}{} &
Labial &
Alveolar &
Postalveolar &
Palatal &
Velar \\

\hline
\multirow{3}{*}{Stop} & voiced &
b & \multicolumn{2}{c|}{d} & & \textipa{g} \textlangle g\textrangle \\

\cline{2-7}
 & voiceless &
p & \multicolumn{2}{c|}{t} & & k \\

\cline{2-7}
 & aspirated &
\raisebox{-1pt}{p\textipa{\super h}\ \textlangle ph\textrangle} & \multicolumn{2}{c|}{\raisebox{-1pt}{t\textipa{\super h}\ \textlangle th\textrangle}} & & \raisebox{-1pt}{k\textipa{\super h}\ \textlangle kh\textrangle} \\

\hline
\multicolumn{2}{|l|}{Nasal} &
m & \multicolumn{2}{c|}{n} & \textltailn\ \textlangle gn\textrangle & \textipa{N}\ \textlangle ng\textrangle \\

\hline
\multicolumn{2}{|l|}{Trill} &
& \multicolumn{2}{c|}{r} & & \\

\hline
\multirow{2}{*}{Fricative} & voiced &
& & \textipa{Z}\ \textlangle zh\textrangle & & \textipa{G}\ \textlangle gh\textrangle \\

\cline{2-7}
& voiceless &
f & s & \textipa{S}\ \textlangle sh\textrangle & & x \textlangle h\textrangle \\

\hline
\multirow{2}{*}{Affricate} & voiced &
& \raisebox{-1.5pt}{\textipa{\t{dz}}} \textlangle dz\textrangle & \raisebox{-1.5pt}{\textipa{\t{dZ}}} \textlangle j\textrangle & & \\

\cline{2-7}
& voiceless &
& \raisebox{-1.5pt}{\textipa{\t{ts}}} \textlangle ts\textrangle & \raisebox{-1.5pt}{\textipa{\t{tS}}} \textlangle ch\textrangle & & \\

\hline
\raisebox{-2pt}{Lateral} & voiced &
& \multicolumn{2}{c|}{\textlyoghlig} & & \\

\cline{2-7}
\raisebox{1pt}{fricative} & voiceless &
& \multicolumn{2}{c|}{\textbeltl} & & \\

\hline
\multicolumn{2}{|l|}{Approximant} &
\textipa{V} & \multicolumn{2}{c|}{} & j \textlangle y\textrangle & w \\

\hline
\multicolumn{2}{|l|}{Lateral approximant} &
& \multicolumn{2}{c|}{l} & & \\

\hline
\multicolumn{2}{|l|}{Click} &
& \multicolumn{2}{c|}{\textipa{\super N|} \textlangle c\textrangle} & & \\

\hline
\end{tabular}
\caption{Phonemic consonants in Ishiculu. Letters in angle brackets represent the phonemes in Ishiculu orthography.}
\end{table}

\begin{figure}
\centering
\begin{vowel}
\putcvowel{i}{1}
\putcvowel{\textipa{E}\ \textlangle e\textrangle }{3}
\putcvowel{a}{4}
\putcvowel{\textipa{O}\ \textlangle o\textrangle }{6}
\putcvowel{\textipa{7}}{7}
\putcvowel{u}{8}
\end{vowel}
\caption{Phonemic vowels in Ishiculu. Letters in angle brackets represent the phonemes in Ishiculu orthography.}
\end{figure}

\section{Syllable structure}
Possible syllable structures in Ishiculu are (C)V and N\textsubscript 1C\textsubscript 1V, where N\textsubscript 1 is a nasal with the same place of articulation as obstruent C\textsubscript 1. Some examples of syllabification are shown in (\ref{ex:phonology:syllabification}).

\begin{exe}
\ex Syllabification in Ishiculu
\begin{xlist}
\ex \textlangle Ishiculu\textrangle \quad /\textipa{i.Si.\super N|u.lu}/ \quad V.CV.CV.CV
\ex \textlangle unjani\textrangle \quad /\textipa{u.\textltailn\t{dZ}a.ni}/ \quad V.NCV.CV
\end{xlist}
\label{ex:phonology:syllabification}
\end{exe}

As a lexically tonal language, it's uncommon for Ishiculu to have no coda. Mandarin and Cantonese both have some simple codas and lexical tones. The lexical tones survived, but the codas did not. On the other hand, Zulu has no coda and only grammatical tones. 

\section{Tones and stress}

\subsection{Tonemes}

Ishiculu has four tones, shown in Table~\ref{table:phonology:tones}. Note that the tones are not indicated in Ishiculu orthography.

\begin{table}[H]
\centering
\begin{tabular}{|l|c|c|c|c|}
\hline
Description & low & high & rising & falling \\
\hline
IPA diacritic & \textipa{\`a} & \textipa{\'a} & \textipa{\v a} & \textipa{\^a} \\
\hline
Tone contour & 11 & 55 & 35 & 51 \\
\hline
\end{tabular}
\caption{Tonemes in Ishiculu.}
\label{table:phonology:tones}
\end{table}

\subsection{Interactions between voiced stops and tones}

Voiced stops in Ishiculu (i.e. /b/, /d/, and /\textipa g/) add a low-tone onset to the normal tone.

\begin{table}[H]
\centering
\begin{tabular}{|l|c|c|c|c|}
\hline
Normal tone & \textipa{\`a} (low) & \textipa{\'a} (high) & \textipa{\v a} (rising) & \textipa{\^a} (falling) \\
\hline
New tone & \textipa{b\`a} (low) & \textipa{b\v a} (rising) & \textipa{b\v a} (rising) & \textipa{b\textrisefall{a}} (rising-falling) \\
\hline
\end{tabular}
\caption{Tone change after voiced stops.}
\end{table}

\subsection{Stress}
The stress of an Ishiculu word falls on the penultimate syllable and results in lengthening of the vowel. Some examples are shown in (\ref{ex:phonology:stress}).
\begin{exe}
\ex
\begin{xlist}
\ex \textlangle Ishiculu\textrangle \quad /\textipa{i.Si.\super N|u.lu}/ \quad [\textipa{iSi\super N|u:lu}]
\ex \textlangle unjani\textrangle \quad /\textipa{u.\textltailn\t{dZ}a.ni}/ \quad [\textipa{u\textltailn\t{dZ}a:ni}]
\end{xlist}
\label{ex:phonology:stress}
\end{exe}
