\chapter{Phonology}

\section[]{Phoneme inventory of Ishiculu\footnote{Pronounced [\textipa{\`IS\'I\super{N}|\v u:l\`u}]}}
\begin{center}

\begin{tabular}{|l|l|c|c|c|c|c|}
\hline
\multicolumn{2}{|c|}{} &
Labial &
Alveolar &
Postalveolar &
Palatal &
Velar \\

\hline
\multirow{3}{*}{Stop} & voiced &
b & \multicolumn{2}{c|}{d} & & \textipa{g} \textlangle g\textrangle \\

\cline{2-7}
 & voiceless &
p & \multicolumn{2}{c|}{t} & & k \\

\cline{2-7}
 & aspirated &
\raisebox{-1pt}{p\textipa{\super h}\ \textlangle ph\textrangle} & \multicolumn{2}{c|}{\raisebox{-1pt}{t\textipa{\super h}\ \textlangle th\textrangle}} & & \raisebox{-1pt}{k\textipa{\super h}\ \textlangle kh\textrangle} \\

\hline
\multicolumn{2}{|l|}{Nasal} &
m & \multicolumn{2}{c|}{n} & \textltailn\ \textlangle gn\textrangle & \textipa{N}\ \textlangle ng\textrangle \\

\hline
\multicolumn{2}{|l|}{Trill} &
& \multicolumn{2}{c|}{r} & & \\

\hline
\multirow{2}{*}{Fricative} & voiced &
& & \textipa{Z}\ \textlangle zh\textrangle & & \textipa{G}\ \textlangle gh\textrangle \\

\cline{2-7}
& voiceless &
f & s & \textipa{S}\ \textlangle sh\textrangle & & x \textlangle h\textrangle \\

\hline
\multirow{2}{*}{Affricate} & voiced &
& \raisebox{-1.5pt}{\textipa{\t{dz}}} \textlangle dz\textrangle & \raisebox{-1.5pt}{\textipa{\t{dZ}}} \textlangle j\textrangle & & \\

\cline{2-7}
& voiceless &
& \raisebox{-1.5pt}{\textipa{\t{ts}}} \textlangle ts\textrangle & \raisebox{-1.5pt}{\textipa{\t{tS}}} \textlangle ch\textrangle & & \\

\hline
\raisebox{-2pt}{Lateral} & voiced &
& \multicolumn{2}{c|}{\textlyoghlig} & & \\

\cline{2-7}
\raisebox{1pt}{fricative} & voiceless &
& \multicolumn{2}{c|}{\textbeltl} & & \\

\hline
\multicolumn{2}{|l|}{Approximant} &
\textipa{V} & \multicolumn{2}{c|}{} & j \textlangle y\textrangle & w \\

\hline
\multicolumn{2}{|l|}{Lateral approximant} &
& \multicolumn{2}{c|}{l} & & \\

\hline
\multicolumn{2}{|l|}{Click} &
& \multicolumn{2}{c|}{\textipa{\super N|} \textlangle c\textrangle} & & \\

\hline
\end{tabular}
\\[10pt]

\begin{vowel}
\putcvowel{i}{1}
\putcvowel{\textipa{E}\ \textlangle e\textrangle }{3}
\putcvowel{a}{4}
\putcvowel{\textipa{O}\ \textlangle o\textrangle }{6}
\putcvowel{\textipa{7}}{7}
\putcvowel{u}{8}
\end{vowel}

\end{center}

\section{Syllable structure}
Possible syllable structures in Ishiculu are (C)V and N\textsubscript 1C\textsubscript 1V, where N\textsubscript 1 is a nasal with the same place of articulation as C\textsubscript 1. For example:
\begin{center}
\begin{tabular}{rcl}
Ishiculu & /\textipa{i.Si.\super N|u.lu}/ & V.CV.CV.CV \\
unjani & /\textipa{u.\textltailn\t{dZ}a.ni}/ & V.NCV.CV \\
\end{tabular}
\end{center}

\section{Tones and stress}

\subsection{Tone categories}

Ishiculu has four tones.

\begin{center}
\begin{tabular}{|l|c|c|c|c|}
\hline
Description & low & high & rising & falling \\
\hline
IPA diacritic & \textipa{\`a} & \textipa{\'a} & \textipa{\v a} & \textipa{\^a} \\
\hline
Tone contour & 11 & 55 & 35 & 51 \\
\hline
\end{tabular}
\end{center}

\subsection{Interactions between voiced stops and tones}

Voiced stops in Ishiculu (i.e. /b/, /d/, and /\textipa g/) add a low-tone onset to the normal tone.

\begin{center}
\begin{tabular}{|l|c|c|c|c|}
\hline
Normal tone & \textipa{\`a} (low) & \textipa{\'a} (high) & \textipa{\v a} (rising) & \textipa{\^a} (falling) \\
\hline
New tone & \textipa{b\`a} (low) & \textipa{b\v a} (rising) & \textipa{b\v a} (rising) & \textipa{b\textrisefall{a}} (rising-falling) \\
\hline
\end{tabular}
\end{center}

\subsection{Stress}
The stress of an Ishiculu word falls on the penultimate syllable and results in lengthening of the vowel. For example:
\begin{center}
\begin{tabular}{rcl}
Ishiculu & /\textipa{i.Si.\super N|u.lu}/ & [\textipa{iSi\super N|u:lu}] \\
unjani & /\textipa{u.\textltailn\t{dZ}a.ni}/ & [\textipa{u\textltailn\t{dZ}a:ni}] \\
\end{tabular}
\end{center}
