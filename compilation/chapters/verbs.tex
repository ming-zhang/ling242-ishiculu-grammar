\chapter{Verbs}

\section{Agreements}

\newthought{Ishiculu exhibits highly synthetic} verbal morphology. In this sense, it is predominantly head-marking. Prefixation and suffixation are responsible for marking the agreements and the tense. Ishiculu is mostly agglutinative in terms of its verb affixation: different grammatical features of the verb are expressed through different affixes, except for the person and the number, which are contained in the same prefix. There are both inflectional and derivational verb affixes in Ishiculu.

The Ishiculu verb is marked according to a nominative-accusative alignment system. The verb stem is inflected based on the person, number, and, for 3rd person nominal phrases, the noun class. The structure of an inflected Ishiculu verb consists of nominative agreement, dative agreement, accusative agreement, stem, derivational suffixes, and tense, in this order. The same set of prefixes, shown in Table~\ref{table:verbs:personal-prefixes}, is used for all three agreements.

\begin{exe}
\ex
\gll \textit{Shi-shiy-ioani} \textipa{nta\textbeltl o} kitabu. \\
\textsc{1pl}-\textsc{cl7.sg}-like \textsc{clf}.7 book(7) \\
\trans `We (will) like the book.'
\ex
\gll Ari pa \textit{ka-uw-ioani} shidzi. \\
Ari not \textsc{cl1.sg}-\textsc{cl4.sg}-like cheese(4) \\
\trans `Ari doesn't like cheese.'
\end{exe}

\begin{table}
\centering
\begin{tabular}{c|c|c|c|c}
\hline
\multirow{2}{*}{} & \multicolumn{2}{c|}{singular} & \multicolumn{2}{c}{plural} \\
\cline{2-5}
 & before C & before V & before C & before V \\
\hline
\hline
1st person & -ngi- & -ngiw- & -shi- & -shiy- \\
\hline
2nd person & -u- & -uw- & -ni-& -niy- \\
\hline
CL 1/2 & \multicolumn{2}{c|}{-ka-} & \multicolumn{2}{c}{-ba-} \\
\hline
CL 3/4 & -u- & -uw- & -i- & -iy- \\
\hline
CL 5/6 & -\textbeltl i- & -\textbeltl iy- & \multicolumn{2}{c}{-a-} \\
\hline
CL 7/8 & -shi- & -shiy- & -\textipa{Z}i- & -\textipa{Z}iy- \\
\hline
\end{tabular}
\caption{Personal agreement prefixes of Ishiculu verbs.}
\label{table:verbs:personal-prefixes}
\end{table}

\section{Tenses}

Ishiculu has two tenses: present and past. The present tense is marked with a \textit{-\o} suffix, and can be used for future events. The past tense is marked with suffix \textit{\textipa{-mbi}}.

\begin{exe}
\ex
\gll Shi-shiy-ioani-\textit{mbi} \textipa{nta\textbeltl o} kitabu. \\
\textsc{1pl}-\textsc{cl7.sg}-like-\textsc{pst} \textsc{clf}.7 book(7) \\
\trans `We used to like the book.'
\end{exe}

\section{Imperative}

The imperative can occur either alone or with an object prefix, using one of the imperative suffixes shown in Table~\ref{table:verbs:imperative}.

\begin{table}
\centering
\begin{tabular}{c|l|l}
\hline
& \multicolumn{1}{c|}{Alone} & \multicolumn{1}{c}{With object} \\
\hline
Singular & \textit{-a} & \textit{-e} \\
\hline
Plural & \textit{-ani} & \textit{-eni} \\
\hline
\end{tabular}
\caption{Imperative suffixes in Ishiculu.}
\label{table:verbs:imperative}
\end{table}

\begin{exe}
\ex
\gll So-a! \\
read-\textsc{imp} \\
\trans `Read!'
\ex
\gll Shi-so-e \textipa{nta\textbeltl o} kitabu! \\
\textsc{cl7.sg}-read-\textsc{imp} \textsc{clf}.7 book(7) \\
\trans `Read the book!'
\end{exe}

\section{Valency-changing processes}
\subsection{Passive}

Morphological passive in Ishiculu is marked with \textit{\textipa{-wal7}} and decreases the valency of transitive verbs.

\begin{exe}
\ex
\gll U {-bon\textramshorns} -\textit{\textipa{wal7}} -mbi. \\
\textsc{2sg} -see -\textsc{pass} -\textsc{pst}  \\
\trans `You were seen.'
\ex
\gll Nta\textbeltl o kitabu shi-so-\textit{\textipa{wal7}}-mbi. \\
\textsc{clf}.7 book(7) \textsc{cl7.sg}-read-\textsc{pass}-\textsc{pst}  \\
\trans `The book was read.'
\end{exe}

\subsection{Impersonal}
When the verb is impersonal, it does not bear a subject concord prefix.

Ishiculu allows for impersonal meteorological predicates.

\begin{exe}
\ex
\gll Cofi ha-mbi. \\
yesterday rain-\textsc{pst} \\
\trans `It rained yesterday.'
\ex
\gll T\textipa{um7} \textipa{r7}. \\
tomorrow hot \\
\trans `It will be hot tomorrow.'
\end{exe}

Impersonal construction is also used for indefinite subjects.

\begin{exe}
\ex
\gll Pa ngi-shi-so nta\textbeltl o kitabu. \\
not \textsc{1sg}-\textsc{7}-read \textsc{clf}.7 book \\
\trans `I don't read the book.'
\ex
\gll Pa shi-so nta\textbeltl o kitabu. \\
not \textsc{7}-read \textsc{clf}.7 book \\
\trans `One doesn't/shouldn't read the book.'
\end{exe}

\subsection{Causative}

(


The suffix \textit{-\textbeltl a} mark the causative form of a verb. Accordingly, the causee of the causative verb is marked with an indirect object suffix after the causative suffix.

\begin{exe}
\ex
\gll Fu-Philadelphia ni-hi-\textipa{\textbeltl}a-ki Ari \textipa{nta\textbeltl o} kitabu. \\
in-Philadelphia \textsc{1sg}-hear-\textsc{caus}-\textsc{3sg} Ari \textsc{clf}.7 book(7) \\
\trans `I (will) read the book to Ari.'
\end{exe}

)


The suffix \textit{-\textbeltl a} marks the causative form of a verb. The causer of the causative verb is marked with an additional prefix.

\begin{exe}
\ex
\gll Ngi-ka-shi-so-\textipa{\textbeltl}a Ari \textipa{nta\textbeltl o} kitabu. \\
\textsc{1sg}-\textsc{1}-\textsc{7}-have-\textsc{caus} Ari \textsc{clf}.7 book(7) \\
\trans `I (will) make Ari read the book.'
\ex
\gll Fu-Philadelphia ngi-ka-shi-ke-\textipa{\textbeltl}a Ari \textipa{nta\textbeltl o} kitabu. \\
in-Philadelphia \textsc{1sg}-\textsc{1}-\textsc{7}-have-\textsc{caus} Ari \textsc{clf}.7 book(7) \\
\trans `I (will) give the book to Ari in Philadelphia.'
\end{exe}


The suffix \textit{-\textbeltl a} marks the causative form of a verb. The causer of the causative verb is marked with an additional prefix on the verb.

\begin{exe}
\ex
\gll Ngi-ka-shi-so-\textipa{\textbeltl}a Ari \textipa{nta\textbeltl o} kitabu. \\
\textsc{1sg}-\textsc{1}-\textsc{7}-have-\textsc{caus} Ari \textsc{clf}.7 book(7) \\
\trans `I (will) make Ari read the book.'
\end{exe}

\subsubsection{Ditransitive}

Prototypical verbs in Ishiculu are transitive at most, and verbs that semantically require three thematic core arguments are usually expressed through causatives of transitive verbs.

\begin{exe}
\ex
\gll Fu-Philadelphia ngi-ka-shi-ke-\textipa{\textbeltl}a Ari \textipa{nta\textbeltl o} kitabu. \\
in-Philadelphia \textsc{1sg}-\textsc{1}-\textsc{7}-have-\textsc{caus} Ari \textsc{clf}.7 book(7) \\
\trans `I (will) give Ari the book in Philadelphia.'
\ex
\gll {W\textramshorns} babamama ba-ngiw-u-nde-\textipa{\textbeltl}a-mbi mina shizi. \\
\textsc{clf.2} parents(2) \textsc{2.pl}-\textsc{1sg}-\textsc{4.sg}-eat-\textsc{caus}-\textsc{pst} \textsc{1sg} cheese(4) \\
\trans `My parents fed me cheese.'
\end{exe}

\subsection{Passive of causative}

The causee or the object can be promoted to the subject position with the passive construction. Following the parallelism between ditransitive and causative verbs, their passive constructions are also indistinguishable in morphology.

\subsubsection{Promoting causee}

This refers to the passive construction similar to the following:

\begin{exe}
\ex Passive of causative of intransitive in English \\
John makes me cringe. $\rightarrow$ I am made to cringe (by John).
\ex Passive of causative of transitive in English \\
John makes the computer change the date. $\rightarrow$ The computer is made to change the date (by John).
\ex Passive of ditransitive in English \\
John gives the teacher the book. $\rightarrow$ The teacher is given the book (by John).
\end{exe}

This construction in Ishiculu is formed by a passive suffix \textit{-\textipa{wal7}} after the causative suffix. The personal suffix of the causer on the verb is then dropped.

\begin{exe}
\ex Passive of causative of intransitive in Ishiculu
\begin{xlist}
\ex
\gll Ngi-ka -hulu-\textbeltl a-mbi John. \\
\textsc{1sg}-\textsc{1.sg} -cry-\textsc{caus}-\textsc{pst} John \\
\trans `I made John cry.'
\ex
\gll John ka -hulu-\textbeltl a-wal\textramshorns-mbi. \\
John \textsc{1.sg} -cry-\textsc{caus}-\textsc{pass}-\textsc{pst} \\
\trans `John was made to cry.'
\end{xlist}
\ex Passive of ditransitive/causative of transitive in Ishiculu
\begin{xlist}
\ex
\gll Shiy-u-ko -me\textbeltl i-\textbeltl a-mbi hehe. \\
\textsc{1pl}-\textsc{2sg}-\textsc{5.sg} -receive\_by\_mail-\textsc{caus}-\textsc{pst} cake(5) \\
\trans `We mailed you cake.'
\ex
\gll U-ko -me\textbeltl i-\textbeltl a-wal\textramshorns-mbi hehe. \\
\textsc{2sg}-\textsc{5.sg} -receive\_by\_mail-\textsc{caus}-\textsc{pass}-\textsc{pst} cake(5) \\
\trans `You were mailed cake.'
\end{xlist}
\end{exe}

\subsubsection{Promoting object}

There is no passivization of a causative verb that directly promotes the underlying object, but it is possible to passivize the underlying verb, promoting the object to the causee position before promoting it further to the subject position of the matrix clause. As the subject in the active voice becomes impossible to be salient in the passive, the causee becomes ineffable in this passive construction.

\begin{exe}
\ex
\begin{xlist}
\ex Active voice
\gll John ka-u-shi -ke-\textbeltl a meyi sh\textramshorns nta\textbeltl o kitabu. \\
John \textsc{1.sg}-\textsc{3.sg}-\textsc{7.sg} -have-\textsc{caus} \textsc{clf.3} teacher(3) \textsc{clf.7} book(7) \\
\trans `John gives the teacher the book.'
\ex Impossibility of direct promotion of object
\gll * Nta\textbeltl o kitabu shi-u {-ke-\textbeltl a-wal\textramshorns} meyi sh\textramshorns. \\
{} \textsc{clf.7} book(7) \textsc{7.sg}-\textsc{3.sg} -have-\textsc{caus}-\textsc{pass} \textsc{clf.3} teacher(3) \\
\trans Intended meaning: `The book is given to the teacher.'
\ex Causative of passive
\gll John ka-u-shi -ke-wal\textramshorns-\textbeltl a nta\textbeltl o kitabu. \\
John \textsc{1.sg}-\textsc{3.sg}-\textsc{7.sg} -have-\textsc{pass}-\textsc{caus} \textsc{clf.7} book(7) \\
\trans `John gives the book.' (lit. `John makes the book be given.')
\ex Passive of causative of passive
\gll Nta\textbeltl o kitabu shi-ke-wal\textramshorns-\textbeltl a-wal\textramshorns. \\
\textsc{clf.7} book(7) \textsc{7.sg}-have-\textsc{pass}-\textsc{caus}-\textsc{pass} \\
\trans `The book is given.' (lit. `The book is made to be had.')
\ex Ineffability of the causee
\gll * Nta\textbeltl o kitabu shi-u {-ke-wal\textramshorns-\textbeltl a-wal\textramshorns} meyi sh\textramshorns. \\
{} \textsc{clf.7} book(7) \textsc{7.sg}-\textsc{3.sg} -have-\textsc{pass}-\textsc{caus}-\textsc{pass} \textsc{clf.3} teacher(3) \\
\trans Intended meaning: `The book is given to the teacher.' (lit. `The book is made to be had.')
\end{xlist}
\end{exe}

