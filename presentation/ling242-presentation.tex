% Copyright 2004 by Till Tantau <tantau@users.sourceforge.net>.
%
% In principle, this file can be redistributed and/or modified under
% the terms of the GNU Public License, version 2.
%
% However, this file is supposed to be a template to be modified
% for your own needs. For this reason, if you use this file as a
% template and not specifically distribute it as part of a another
% package/program, I grant the extra permission to freely copy and
% modify this file as you see fit and even to delete this copyright
% notice. 

\documentclass{beamer}
% Replace the \documentclass declaration above
% with the following two lines to typeset your 
% lecture notes as a handout:
%\documentclass{article}
%\usepackage{beamerarticle}

\usepackage{textcomp}
\usepackage{multirow}

\usepackage{tipa,vowel,gb4e}
\let\ipa\textipa
\def\asp{\ipa{\super h}}
\let\eachwordone=\normalfont
\let\eachwordtwo=\normalfont

\usepackage{xcolor}

\usepackage{tikz}

\newcommand{\exsc}[1]{{\tiny\MakeUppercase{#1}}}

% There are many different themes available for Beamer. A comprehensive
% list with examples is given here:
% http://deic.uab.es/~iblanes/beamer_gallery/index_by_theme.html
% You can uncomment the themes below if you would like to use a different
% one:
%\usetheme{AnnArbor}
%\usetheme{Antibes}
%\usetheme{Bergen}
%\usetheme{Berkeley}
%\usetheme{Berlin}
%\usetheme{Boadilla}
%\usetheme{boxes}
%\usetheme{CambridgeUS}
%\usetheme{Copenhagen}
%\usetheme{Darmstadt}
\usetheme{default}
%\usetheme{Frankfurt}
%\usetheme{Goettingen}
%\usetheme{Hannover}
%\usetheme{Ilmenau}
%\usetheme{JuanLesPins}
%\usetheme{Luebeck}
%\usetheme{Madrid}
%\usetheme{Malmoe}
%\usetheme{Marburg}
%\usetheme{Montpellier}
%\usetheme{PaloAlto}
%\usetheme{Pittsburgh}
%\usetheme{Rochester}
%\usetheme{Singapore}
%\usetheme{Szeged}
%\usetheme{Warsaw}

\title{An Overview of Ishiculu Grammar}

% A subtitle is optional and this may be deleted
%\subtitle{Optional Subtitle}

\author{Ming Zhang}
% - Give the names in the same order as the appear in the paper.
% - Use the \inst{?} command only if the authors have different
%   affiliation.

\institute[University of Pennsylvania] % (optional, but mostly needed){}
% - Use the \inst command only if there are several affiliations.
% - Keep it simple, no one is interested in your street address.

\date{December 5, 2017}
% - Either use conference name or its abbreviation.
% - Not really informative to the audience, more for people (including
%   yourself) who are reading the slides online

\subject{Ishiculu Grammar}
% This is only inserted into the PDF information catalog. Can be left
% out. 

% If you have a file called "university-logo-filename.xxx", where xxx
% is a graphic format that can be processed by latex or pdflatex,
% resp., then you can add a logo as follows:

% \pgfdeclareimage[height=0.5cm]{university-logo}{university-logo-filename}
% \logo{\pgfuseimage{university-logo}}

% Delete this, if you do not want the table of contents to pop up at
% the beginning of each subsection:
%\AtBeginSubsection[]
%{
%  \begin{frame}<beamer>{Outline}
%    \tableofcontents[currentsection,currentsubsection]
%  \end{frame}
%}

% Let's get started
\begin{document}

\begin{frame}
  \titlepage
\end{frame}

%\begin{frame}{Outline}
%  \tableofcontents
%  % You might wish to add the option [pausesections]
%\end{frame}

\begin{frame}{Phonology}
\begin{itemize}
\item 31 consonants, including: \ipa{\super N|}; no uvular/glottal; \ipa{d}/\ipa{t}/\ipa{t}\asp; no v/z
\item 
Six vowels, with ``moderately high'' consonant-vowel ratio
\begin{vowel}
\putcvowel{i}{1}
\putcvowel{\ipa{E}\ \textlangle e\textrangle }{3}
\putcvowel{a}{4}
\putcvowel{\ipa{O}\ \textlangle o\textrangle }{6}
\putcvowel{\ipa{7}}{7}
\putcvowel{u}{8}
\end{vowel}
\pause
\item Syllable structure: either (C)V or N\textsubscript{1}C\textsubscript{1}V
\pause
\item Tonal with four tonemes
\end{itemize}
\end{frame}

%\subsection{Second Subsection}

% You can reveal the parts of a slide one at a time
% with the \pause command:
\begin{frame}{Noun classes; morphology; word order}
  \begin{itemize}
  \item Eight noun classes
  \pause
  \item Synthetic verbal morphology; no overt case morphology on pro/nouns
  \begin{itemize}
  \item Nom. - dat. - acc. - \textsurd\ - derivational - tense
  \end{itemize}
  {\scriptsize
    \begin{exe}
    \sn
    \gll \textcolor{blue}{Bona} ba-ngiw-i -nde-\textbeltl a-mbi \textcolor{blue}{mina} nta u. \textcolor{blue}{Mina} ngi-ba-i -nde-\textbeltl a-mbi \textcolor{blue}{bona} nta u. \\
    \exsc{pron.1.pl} \exsc{1.pl}-\exsc{1sg}-\exsc{4.pl} -eat-\exsc{caus}-\exsc{pst} \exsc{pron.1sg} three \exsc{clf.4} \exsc{pron.1sg} \exsc{1sg}-\exsc{1.pl}-\exsc{4.pl} -eat-\exsc{caus}-\exsc{pst} \exsc{pron.1.pl} three \exsc{clf.4} \\
    \trans `They fed me three (blocks of cheese, say). I fed them three.'
    \end{exe}
  }
\pause
\item Basic word order SVO
\item Topicalization of the object
{\scriptsize
\begin{exe}
\sn
\gll {Ntoch\textramshorns} ngi-zhi-ke (nta) \textcolor{blue}{a\textlyoghlig o}. \\
cars(8) \exsc{1sg}-\exsc{8.pl}-have three \exsc{clf.8} \\
\trans `Cars, I have (three).'
\end{exe}
}
\pause
\item Locative expressions
{\scriptsize
\begin{exe}
\sn
\gll Ts\textramshorns-nji-tso \textbeltl i-huma i\textlyoghlig i ika. \\
\exsc{prep.sur}-table-\exsc{post.abl} \exsc{3sg}-jump \exsc{clf}.6 cat(6) \\
\trans `Off the table jumps a cat.'
\end{exe}
}
\end{itemize}
\end{frame}

\begin{frame}{Classifiers}

{\scriptsize
\begin{exe}
\sn
\gll Ngiw-u-nde-mbi shidzi. \\
\exsc{1sg}-\exsc{4.sg}-eat-\exsc{pst} cheese(4) \\
\trans `I ate cheese.'
\sn
\gll Ngi-zhiy-ioani kitabu. \\
\exsc{1sg}-\exsc{7.pl}-like book(7) \\
\trans `I like books.'
\end{exe}
}
\pause
{\scriptsize
\begin{exe}
\sn
\gll Ngi-zhi-ke \textcolor{red}{nta} \textcolor{blue}{nta\textbeltl o} kitabu. \\
\exsc{1sg}-\exsc{7.pl}-have three \exsc{clf.7} book(7) \\
\trans `I have three books.'
\end{exe}
}
\pause
{\scriptsize
\begin{exe}
\sn
\gll Ngiw-u-nde-mbi \textcolor{blue}{u} shidzi \textcolor{red}{ce-John}. \\
\exsc{1sg}-\exsc{4.sg}-eat-\exsc{pst} \exsc{clf.4} cheese(4) \exsc{gen}-John \\
\trans `I ate John's cheese.'
\sn
\gll Ngiw-u-nde-mbi \textcolor{red}{ligha} \textcolor{blue}{u} shidzi. \\
\exsc{1sg}-\exsc{4.sg}-eat-\exsc{pst} this \exsc{clf.4} cheese(4) \\
\trans `I ate this cheese.'
\sn
\gll Ngiw-u-nde-mbi \textcolor{blue}{u} shidzi \textcolor{red}{uw-uw-ioani} \textcolor{red}{wona}. \\
\exsc{1sg}-\exsc{4.sg}-eat-\exsc{pst} \exsc{clf.4} cheese(4) \exsc{2sg}-\exsc{4.sg}-like \exsc{prn.4.sg} \\
\trans `I ate the cheese that you like.'
\end{exe}
}
\end{frame}

\begin{frame}{Valence changing}
\begin{itemize}
\item Causative
{\scriptsize
\begin{exe}
\sn
\gll Ngi-ka-shi-so-\textcolor{blue}{\textbeltl a} Ari \ipa{nta\textbeltl o} kitabu. \\
\exsc{1sg}-\exsc{1.sg}-\exsc{7.sg}-read-\exsc{caus} Ari \exsc{clf}.7 book(7) \\
\trans `I (will) make Ari read the book.'
\end{exe}
}
\pause
\begin{itemize}
\item Prototypical verbs in Ishiculu are transitive at most.
{\scriptsize
\begin{exe}
\sn
\gll Fu-Philadelphia ngi-ka-shi-ke-\textcolor{blue}{\textbeltl a}-mbi Ari \ipa{nta\textbeltl o} kitabu. \\
in-Philadelphia \exsc{1sg}-\exsc{1.sg}-\exsc{7.sg}-have-\exsc{caus}-\exsc{pst} Ari \exsc{clf}.7 book(7) \\
\trans `I gave Ari the book in Philadelphia.'
\end{exe}
}
\end{itemize}
\pause
\item Passive
{\scriptsize
\begin{exe}
\sn
\gll Nta\textbeltl o kitabu shi-so-\textcolor{blue}{\ipa{wal7}}-mbi. \\
\exsc{clf}.7 book(7) \exsc{7.sg}-read-\exsc{pass}-\exsc{pst}  \\
\trans `The book was read.'
\end{exe}
}
\pause
\begin{itemize}
\item
The agent is then ineffable.
{\scriptsize
\begin{exe}
\sn
\gll * Nta\textbeltl o kitabu shi-ngi-so-{\ipa{wal7}}-mbi mina. \\
{} \exsc{clf}.7 book(7) \exsc{7.sg}-\exsc{1sg}-read-\exsc{pass}-\exsc{pst} \exsc{pron.1sg} \\
\trans Intended meaning: `The book was read by me.'
\end{exe}
}
\end{itemize}
\end{itemize}
\end{frame}

\begin{frame}{Passive of causative}
\begin{itemize}
\item Causee is promoted directly.
{\scriptsize
\begin{exe}
\sn
\gll U-ko -me\textbeltl i-\textcolor{blue}{\textbeltl a-wal\textramshorns}-mbi ishi hehe. \\
\exsc{2sg}-\exsc{5.sg} -receive-\exsc{caus}-\exsc{pass}-\exsc{pst} \exsc{clf.5} cake(5) \\
\trans `You were sent a cake.'
\end{exe}
}
\pause
\item Object needs to be promoted to the causee position first.
{\scriptsize
\begin{exe}
\sn
\gll Nta\textbeltl o kitabu shi-ke-\textcolor{blue}{wal\textramshorns-\textbeltl a-wal\textramshorns}. \\
\exsc{clf.7} book(7) \exsc{7.sg}-have-\exsc{pass}-\exsc{caus}-\exsc{pass} \\
\trans `The book is given.'
\end{exe}
}
\end{itemize}
\end{frame}

\begin{frame}{Relative clauses}

Primary strategy relativizes subject
{\scriptsize
\begin{exe}
\sn
\gll Ngiw-uw-ioani meyi {sh\textramshorns} u-ka-shi-ke-\textbeltl a-mbi Ari nta\textbeltl o kitabu. \\
{\tiny 1SG}-{\tiny 3.SG}-like {\tiny CLF.3} teacher(3) {\tiny 3.SG}-{\tiny 1.SG}-{\tiny 7.SG}-have-{\tiny CAUS}-{\tiny PST} Ari {\tiny CLF.7} book(7) \\
\trans `I like the teacher that gave Ari the book.'
\end{exe}
}
\pause
Direct objects need to be passivized
{\scriptsize
\begin{exe}
\sn
\gll * Ngiw-shiy-ioani nta\textbeltl o kitabu meyi {sh\textramshorns} u-so-mbi. \\
{} {\tiny 1SG}-{\tiny 7.SG}-like {\tiny CLF.7} book(7) {\tiny CLF.3} teacher(3) {\tiny 3.SG}-read-{\tiny PST} \\
\trans Intended meaning: `I like the book that the teacher read.'
\sn
\gll Ngiw-shiy-ioani nta\textbeltl o kitabu u-so-wal\textramshorns-mbi. \\
{\tiny 1SG}-{\tiny 7.SG}-like {\tiny CLF.7} book(7) {\tiny 3.SG}-read-{\tiny PASS}-{\tiny PST} \\
\trans `I like the book that was read.'
\end{exe}
}
Resumptive pronouns
{\scriptsize
\begin{exe}
\sn
\gll Ngiw-shiy-ioani nta\textbeltl o kitabu meyi {sh\textramshorns} u-so-mbi \textcolor{blue}{sona}. \\
{\tiny 1SG}-{\tiny 7.SG}-like {\tiny CLF.7} book(7) {\tiny CLF.3} teacher(3) {\tiny 3.SG}-read-{\tiny PST} {\tiny PRN.7.SG} \\
\trans `I like the book that the teacher read.'
\end{exe}
}
\end{frame}

\begin{frame}[c]
\Huge Questions? Comments?
\end{frame}

\end{document}



\begin{frame}{History}
\begin{figure}
\begin{tikzpicture} 
    \node[anchor=south west,inner sep=0] (image) at (0,0) {\includegraphics[width=1\textwidth]{map.png}};
    \begin{scope}[x={(image.south east)},y={(image.north west)}]
        \draw[->,red,ultra thick] (0.755,0.605) to [out=180,in=60] (0.49,0.22);
    \end{scope}
\end{tikzpicture}
\caption{Migration of Chinese people to South Africa in the 2030s}
\end{figure}
\end{frame}

\begin{frame}{Consonant inventory}

\begin{tabular}{|l|l|c|c|c|c|c|}
\hline
\multicolumn{2}{|c|}{} &
Lab. &
Alv. &
P. alv. &
Pal. &
Vel. \\

\hline
\multirow{3}{*}{Stop} & voiced &
b & \multicolumn{2}{c|}{d} & & \ipa{g} \textlangle g\textrangle \\

\cline{2-7}
 & voiceless &
p & \multicolumn{2}{c|}{t} & & k \\

\cline{2-7}
 & \textcolor{blue}{aspirated} &
\raisebox{-1pt}{p\asp\ \textlangle ph\textrangle} & \multicolumn{2}{c|}{\raisebox{-1pt}{t\asp\ \textlangle th\textrangle}} & & \raisebox{-1pt}{k\asp\ \textlangle kh\textrangle} \\

\hline
\multicolumn{2}{|l|}{Nasal} &
m & \multicolumn{2}{c|}{n} & \textltailn\ \textlangle gn\textrangle & \ipa{N}\ \textlangle ng\textrangle \\

\hline
\multicolumn{2}{|l|}{Trill} &
& \multicolumn{2}{c|}{r} & & \\

\hline
\multirow{2}{*}{Fricative} & voiced &
& & \ipa{Z}\ \textlangle zh\textrangle & & \ipa{G}\ \textlangle gh\textrangle \\

\cline{2-7}
& voiceless &
f & s & \ipa{S}\ \textlangle sh\textrangle & & x \textlangle h\textrangle \\

\hline
\multirow{2}{*}{Affricate} & voiced &
& \raisebox{-1.5pt}{\ipa{\t{dz}}} \textlangle dz\textrangle & \raisebox{-1.5pt}{\ipa{\t{dZ}}} \textlangle j\textrangle & & \\

\cline{2-7}
& voiceless &
& \raisebox{-1.5pt}{\ipa{\t{ts}}} \textlangle ts\textrangle & \raisebox{-1.5pt}{\ipa{\t{tS}}} \textlangle ch\textrangle & & \\

\hline
\raisebox{-2pt}{Lateral} & voiced &
& \multicolumn{2}{c|}{\textlyoghlig} & & \\

\cline{2-7}
\raisebox{1pt}{fricative} & voiceless &
& \multicolumn{2}{c|}{\textbeltl} & & \\

\hline
\multicolumn{2}{|l|}{Approximant} &
\ipa{V} & \multicolumn{2}{c|}{} & j \textlangle y\textrangle & w \\

\hline
\multicolumn{2}{|l|}{Lateral approximant} &
& \multicolumn{2}{c|}{l} & & \\

\hline
\multicolumn{2}{|l|}{Click} &
& \multicolumn{2}{c|}{\textcolor{blue}{\ipa{\super N|}} \textlangle c\textrangle} & & \\

\hline
\end{tabular}

\end{frame}
